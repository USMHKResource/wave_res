\section{Background}
\label{sec:background}

The methodology described herein and the methodology defined in IEC 62600-101 occupy distinct and complimentary branches of resource assessments:
\begin{enumerate}
\item {\bf \em Regional resource assessments} seek to quantify the total power available over the region of interest in units of power (e.g., kW, GW, or TWh/yr). 
They are of value to policy makers who need a direct and clear assessment of the magnitude of resource in their region for comparison to other resource options, or for comparison to other regions. They are typically calculated over relatively large areas (e.g., O(1,000 km$^2$) all the way up to global scales). The EPRI 2011 report is a regional resource assessment for wave energy \citep[][]{EPRIwaveresource2011} but several others have been conducted \citep[e.g., ][]{gunnQuantifyingGlobalWave2012,regueroGlobalWavePower2015,motaWaveEnergyPotential2014}.
\item {\bf \em Site resource assessments} provide the resource statistics that are useful for evaluating a site’s power production potential. They quantify the resource opportunity in terms of resource statistics such as wave height, wave period, wave direction, omni-directional wave power, the joint-probability distribution of wave-height and wave-period, and the directional wave spectrum. They are of value to project developers who use the statistics to quantify the opportunity for developing a project at that specific site. The IEC 62600-101 technical specification provides a detailed methodology for conducting site assessments, computing these statistics, and it also provides a consistent format for presenting the results. By using the IEC 62600-101, developers and other parties can compare sites and projects efficiently and with confidence in the accuracy and consistency of the underlying methods. Ultimately, this should improve investor and financier confidence, and thereby reduce project costs. Several works follow the IEC methodology \citep[e.g., ][]{zhengAssessingChinaSea2013, neillWavePowerVariability2013, sierraWaveEnergyResource2013, robertsonCharacterizingShoreWave2014, yangCharacteristicsVariabilityNearshore2020, lokuliyana_sri_2020, garcia-medina_wave_2021}.
\end{enumerate}

It is worth noting that these two branches of resource assessment are different from the three `classes' of resource assessment defined in IEC 62600-101 (reconaissance, feasibility, and design). Those classes are defined primarily in terms of level-of-accuracy (low, medium, and high, respectively) at successively smaller spatial scales, and they are all essentially site assessments in the sense that they generate the statistics of interest to project developers.
The IEC 62600-101 standard does not provide definitive guidance on regional resource assessment, but it does include the language ``the effects of [WEC arrays] on wave propagation should be included in the numerical model''. This note provides a basis for incorporating wave directionality into site assessments, which is a critical aspect of regional assessments like this one.

Another important distinction within the field of resource assessment is related to the technology considerations and spatial constraints of resource assessments. In this space IEC TC114 provides clear and informative definitions \citep{internationalelectrotechnicalcommissionPartTerminologyEdition2020}:
\begin{itemize}
  \item the {\it theoretical resource} is the ``energy available in the resource''
  \item the {\it technical resource} is the ``proportion of the theoretical resource that can be captured using existing technology options without consideration of external constraints''
  \item the {\it practical resource} is the ``proportion of the technical resource that is available after consideration of external constraints''
\end{itemize}

The theoretical resource is a basis for the other two flavors of resource assessment, as it indicates the total power that could be extracted from the resource according to laws of physics and assuming perfectly efficient technology. In the case of wave energy, the energy {\em available} is equivalent to the energy {\em contained in} the wave field because it is theoretically possible to extract all of the energy from the waves as originally demonstrated by the ``Salter Duck'' \citep{salterRecentProgressDucks1980}. This is in contrast to the ``Betz limit'' of wind energy, which states that a wind turbine cannot extract more than 59.3 \% of the kinetic energy in the wind.

Real technologies, of course, are not perfectly efficient. Wave energy converters often reflect some wave energy, and generate some amount of turbulence in the vicinity of the device. There are also other losses to heat such those associated with converting mechanical energy to electricity and transmitting that electricity to the grid. An estimate of the ``technical resource'' accounts for these factors in order to provide decision makers (whether they are project developers, or policy makers) an estimate of the opportunity in terms of real technologies available today. In site resource assessments, this is typically accomplished by combining the joint-probability distribution (e.g., from IEC 62600-101) with a specific technology's power matrix (e.g., from IEC 62600-100) to estimate a project's annual energy production (AEP, in units of power). EPRI 2011 took the regional estimate of the U.S. technical resource based on reasonable assumptions about device conversion efficiency and device spacing.

The practical resource takes another step in considering what is realistic by accounting for `external constraints'. These constraints account for the realistic ocean and land-use factors that cannot be quantified purely in terms of physical or technical constrains, and indluce: permitting, alternate ocean uses, and other types of restrictions that limit the placement and energy extraction of projects. At regional scales, practical resource assessment often involves ad-hoc ``fraction of the technical resource'' assumptions. This is because a detailed accounting of external constraints at regional scales is a large undertaking, and is particularly challenging for technologies—such as wave energy—that are still in their infancy. At the site-specific scale, a project is often designed to maximize the energy production at a site within the external constraints that exist there, which is a defacto practical resource assessment.

Finally, there is also an effort to categorize wave energy resource statistics into `wave resource classes' in order to support and inform a framework for device design standards. Within this framework, wave devices could be certified to have been designed and tested to standards that align with the resource class(es) present in the technology developers target market \citep[][]{ahnGlobalWaveEnergy2022,nearyClassificationSystemsWave2018}.
