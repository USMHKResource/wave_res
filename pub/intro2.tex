

%--------------- SNIP ----------------

Furthermore, discussions with several leading wave energy experts revealed two additional key criticisms of existing wave resource assessment methods: 1) that the domain of the resource assessments are often arbitrary and are not necessarily consistent with relevant political boundaries, and 2) that the methods only account for wave energy arriving at the edge of the boundary and not for waves generated within the boundary.

These lingering critiques have resulted in a “define your own method” approach to regional wave resource assessment. Some works follow the EPRI 2011 methodology , while others do account for directionality, but most assessments at large scales avoid the issue resulting in a reconnaissance-level assessment that does not directly quantify total power available across the region . This makes it difficult to make apples-to-apples comparisons, puts the industry’s credibility at risk, and does not provide policy makers with the kind of information they need. 

The primary objective of this work is to address the critiques and questions in the previous paragraphs directly, and thereby bring clarity to the topic of region-scale theoretical resource assessment. In time, we hope that this clarity serves as a basis for consensus on the subject so that researchers can deliver assessments that are consistent, and then move on to other important research areas – including technical and practical resource assessment. The secondary objective of this work is to provide a refined estimate of the U.S. wave resource based on this new method and updated model outputs.

In the case of wave energy, the National Academy review points out that
because the method used did not account for wave direction, it ``has the
potential to double-count a portion of the wave energy''. Furthermore, discussion with several leading wave energy experts reveals two additional key criticisms of existing wave resource assessment methods: 1) that the domain of the resource assessments are often arbitrary in that they are not necessarily consistent with the relevant political boundaries that they are meant to be applicable to, and 2) that the methods only account for wave energy arriving at the edge of the boundary and do not account for waves generated within the boundary itself.

These lingering critiques have resulted in a ``define your own method'' approach to regional wave resource assessment. Some works follow the EPRI 2011 methodology \citep[e.g., ]{kumarWaveEnergyResource2015}, while others account for directionality \citep[e.g., ]{gunnQuantifyingGlobalWave2012, regueroGlobalWavePower2015, hemerRevisedAssessmentAustralia2017}, but most assessments at large scales seem to avoid the issue and instead deliver a reconnaissance-level assessment that does not directly quantify total power available across the region \citep[e.g.,]{robertsonCharacterizingShoreWave2014, sierraWaveEnergyResource2013, zhengAssessingChinaSea2013, neillWavePowerVariability2013, alonsoWaveEnergyResource2015}.
This makes it difficult to make apples-to-apples comparisons, puts the industry's credibility at risks, and -- perhaps most importantly -- doesn't provide policy makers with the kind of information they need. The primary objective of this work is to address the critiques and questions in the previous paragraph directly, and thereby bring clarity to the topic of region-scale theoretical resource assessment. In time, we hope that this clarity serves as a basis for consensus on the subject so that researchers can deliver assessments that are consistent, and then move on to other important research areas -- including technical and practical resource assessment. The secondary objective of this work is to provide a refined estimate of
the U.S. wave resource based on this new method, and new model
results. 

These objectives are accomplished by first discussing the details of regional theoretical resource assessment and presenting our proposed approach (section 2). In section 3, we present the results of this approach applied to each region of the U.S. coastline in comparison to alternate methods that have been used. Section 4 discusses: a) the methodology compared to alternate approaches, b) the regional resource in the context of existing energy markets, and c) the predictability of wave energy. We conclude (section 5) with a summary of results, the proposed methodology, and a view toward how to unify wave energy site and resource assessment.