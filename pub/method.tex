\section{Method}

The overarching goal of this study is to define the wave energy resources and to
develop a consistent and accurate methodology to estimate the total
theoretically available wave resource for a given region. The wave climate is
simulated by implementing WAVEWATCH III® v5.16
\citep{tolmanDistributedmemoryConceptsWave2002} in all model
simulations in this study. WAVEWATCH III® (hereafter WW3) is a skillful and well
established numerical model that has been implemented successfully in many
previous wave resource assessments
\citep[e.g.,][]{garcia-medinaWaveResourceAssessment2014,yangWaveModelTest2017}.
WW3 solves the five-dimensional action balance equation:

\begin{align}
    \frac{DN}{Dt} = \frac{\Src{tot}}{\sigma} = \frac{1}{\sigma}\left ( \Src{in} + \Src{ds} + \Src{nl} + \Src{bot} + \Src{brk} \right )
\end{align}
where $N(t,x,y,\sigma,\theta) \equiv D/\sigma$ is the wave action, $D$ is the variance spectrum, $t$ is
time, $x$ and $y$ are the spatial coordinates, $\sigma$ is the radian frequency, and $\theta$ is
the direction of wave propagation. The wave action is conserved in the absence
of sinks and sources of energy, their combined effect is represented as $\Src{tot}$. In
this study we implement the ST4 physics package option in WW3 to simulate wind
energy input ($\Src{in}$) and dissipation due to whitecapping ($\Src{ds}$)
\citep{ardhuinObservationSwellDissipation2009}. The non-linear quadruplet
interactions ($\Src{nl}$) are modeled with the
Hasselmann and Hasselmann
(\citeyear{hasselmannComputationsParameterizationsNonlinear1985}) formulation.
Bottom friction ($\Src{bot}$) and depth induced wave breaking ($\Src{brk}$) are
modeled with the JONSWAP Hasselmann etal.
\citeyear{hasselmannMeasurementsWindwaveGrowth1973} and Battjes
and Janssen \citeyear{battjesEnergyLossSetup1978} models, respectively.
Default parameters are used for all formulations. The wave spectrum is
discretized with 24 equally spaced bins in $\theta$ space and 29 logarithmically spaced
frequency bins from 0.035Hz to 0.5Hz. \note{Details of the spatial grids are
provided... what are we going to say about the spatial grids?}


Calibrated WW3 models, remote and local terms...

\subsection{Model Configuration}

\subsection{Remote Resource}

We utilize EEZ boundaries defined by \citep[]{flandersmarineinstituteMaritimeBoundariesGeodatabase}.

\subsection{Local Resource}

\begin{enumerate}
\item Global WW3 feeds into regional WW3 model (mid-level resolution)
\item ST4 package (improvement in power density and Hs, but overpredicts energy-period<-- is there a citation for this)
\item We add all source terms together (except bottom friction), because:
  \begin{itemize}
  \item Questions about accuracy of individual terms \citep{garcia-medinaWaveResourceAssessment2014}
  \item The length-scale over which the input-term acts is similar to that of the dissipation terms.
  \item The non-linear terms transfer energy to lower frequencies (with some losses), and in general the input-term adds energy at frequencies that is above the frequency that most WECs are expected to operate. \note{Is this true? What about small devices/PBE?}
  \end{itemize}
\item integrate source-terms in direction (direction not important), because within the domain energy can be absorbed from all directions.
\item No currents, so wave action, ‘N’, is wave energy.
\item State number of directions/frequencies (should we compare different direc/freq resolutions)
\item Domain: entire U.S. EEZ (including territories)
\item 30+ year hindcast
\item ‘Control volume’ is EEZ surrounding a continent/island. We break this into pieces for each country. I.e.: we don’t include line-integral at boundary between countries.
\item Save full directional spectra at 10 nmi intervals from mainland, out to 200 nmi. Can we store all of this data? UNITS: W/m
\item Save full directional spectra at 100-m depth intervals from mainland, out to 200 nmi (or 1000 m?) UNITS: W/m
\item Total power available within a specified domain (e.g., the EEZ) = Line integral of wave-flux at boundary + Area integral of source terms
\item Note that wave-fluxes at EEZ boundaries between nations are not included because this would be ‘double counting’.
\end{enumerate}

\subsection{Potential Resource}

\begin{itemize}
\item Basically this is the same as above, but we add an energy extraction sink-term at the EEZ boundary. This does two things:
\item We can compare the local resource with and without remote waves.
\item Global WW3 feeds into regional WW3 model (mid-level resolution)
\end{itemize}

