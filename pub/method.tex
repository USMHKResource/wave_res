\section{Method}

Calibrated WW3 models, remote and local terms...

\subsection{Model Configuration}

\subsection{Remote Resource}

We utilize EEZ boundaries defined by \citep[]{flandersmarineinstituteMaritimeBoundariesGeodatabase}.

\subsection{Local Resource}

\begin{enumerate}
\item Global WW3 feeds into regional WW3 model (mid-level resolution)
\item ST4 package (improvement in power density and Hs, but overpredicts energy-period<-- is there a citation for this)
\item We add all source terms together (except bottom friction), because:
  \begin{itemize}
  \item Questions about accuracy of individual terms \citep{garcia-medinaWaveResourceAssessment2014}
  \item The length-scale over which the input-term acts is similar to that of the dissipation terms.
  \item The non-linear terms transfer energy to lower frequencies (with some losses), and in general the input-term adds energy at frequencies that is above the frequency that most WECs are expected to operate. \note{Is this true? What about small devices/PBE?}
  \end{itemize}
\item integrate source-terms in direction (direction not important), because within the domain energy can be absorbed from all directions.
\item No currents, so wave action, ‘N’, is wave energy.
\item State number of directions/frequencies (should we compare different direc/freq resolutions)
\item Domain: entire U.S. EEZ (including territories)
\item 30+ year hindcast
\item ‘Control volume’ is EEZ surrounding a continent/island. We break this into pieces for each country. I.e.: we don’t include line-integral at boundary between countries.
\item Save full directional spectra at 10 nmi intervals from mainland, out to 200 nmi. Can we store all of this data? UNITS: W/m
\item Save full directional spectra at 100-m depth intervals from mainland, out to 200 nmi (or 1000 m?) UNITS: W/m
\item Total power available within a specified domain (e.g., the EEZ) = Line integral of wave-flux at boundary + Area integral of source terms
\item Note that wave-fluxes at EEZ boundaries between nations are not included because this would be ‘double counting’.
\end{enumerate}

\subsection{Potential Resource}

\begin{itemize}
\item Basically this is the same as above, but we add an energy extraction sink-term at the EEZ boundary. This does two things:
\item We can compare the local resource with and without remote waves.
\item Global WW3 feeds into regional WW3 model (mid-level resolution)
\end{itemize}

