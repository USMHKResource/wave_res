\section{Introduction}

Interest in wave energy has grown considerably over the last decade, and the Executive Committee of Ocean Energy Systems, an intergovernmental collaboration between countries under a framework established by the International Energy Agency (IEA), has set a goal of installing 300GW of ocean energy capacity by 2050 \citep[]{babaritOceanWaveEnergy2017,huckerbyInternationalVisionOcean2017}. Wave energy is seen as particularly valuable for its predictability and reliability on timescales of hours to a few days \citep{parkinsonIntegratingOceanWave2015}, and for the fact that the resource is concentrated along coastlines where large fractions of the world's populations live. \DIFdelbegin \DIFdel{Wave energy has been found to complement other renewable energy sources in a way that enhances the resilience of microgrids, important for islanded communities pursuing carbon free energy sources \mbox{%DIFAUXCMD
\citep[e.g.][]{newman_evaluating_2022}}\hskip0pt%DIFAUXCMD
. }\DIFdelend Furthermore, energy markets beyond grid connectivity are being actively explored internationally. For instance, the U.S. Department of Energy (DOE) under the ``Powering the Blue Economy'' initiative  is supporting the development of technologies that provide power to unique market applications such as desalination, ocean sensing, aquaculture, maritime shipping, and coastal community resiliency \citep{livecchiPoweringBlueEconomy2019}. Similar efforts are being pursued in the European Union as well as Australia and New Zealand through the Blue Economy Cooperative Research Center \citep{europeancommission2021EUBlueEconomy2021}, (https://blueeconomycrc.com.au/).

Developing these markets require detailed knowledge of the wave energy resource. There are two scales at which the resource is described depending on the objective: a) site-focused resource assessments, which provide the data (time-series of wave energy, wave height, wave period, wave direction, etc.) that wave energy project developers need for identifying sites and planning projects \citep[e.g., ][]{internationalelectrotechnicalcommissionPart101Wave2015, robertsonCharacterizingShoreWave2014, kumarWaveEnergyResource2015}, and b) regional resource assessments, which quantify the total wave energy resource across a large area that the public, policy makers, and other stakeholders use to quantify the high-level value proposition \citep[e.g., ][]{EPRIwaveresource2011, gunnQuantifyingGlobalWave2012, hemerRevisedAssessmentAustralia2017}.
\DIFdelbegin \DIFdel{This distinction, although subtle, is important because requires two different methodologies to achieve the resource assessment. 
}\DIFdelend 

A consistent methodology to perform these assessments will be beneficial to the marine energy industry. To that effect, the International Electrotechnical Commission's Technical Committee on Marine Energy (IEC TC-114) has published a robust and consensus-based methodology for \DIFdelbegin \DIFdel{site-focused }\DIFdelend wave resource assessment \citep[][hereafter, 62600-101]{internationalelectrotechnicalcommissionPart101Wave2015}. The IEC TC-114 methodology addresses site-focused resource assessments and has been used in studies around the world, such as Australia \citep{hemerRevisedAssessmentAustralia2017}, Alaska in USA \citep{garciamedinaWaveResourceCharacterization2021}, Ireland \citep{ramosExploringUtilityEffectiveness2016}, Sri Lanka \citep{lokuliyanaSriLankanWave2020}, US East Coast \citep{ahnNearshoreWaveEnergy2021}. However, to the authors’ knowledge no definitive consensus-based methodology for regional wave resource assessments exists. Previous regional resource assessments have utilized disparate methodologies, which can lead to skepticism and confusion. 

In the USA, the Electric Power Research Institute performed the first regional wave resource assessment to quantify the total wave energy available in US waters \citep[][hereafter, EPRI 2011]{EPRIwaveresource2011}. A review study conducted by the National Academy of Sciences \DIFdelbegin \DIFdel{(NAS) }\DIFdelend on U.S. marine energy resource assessments found that \DIFaddbegin \DIFadd{while in general }\DIFaddend the resource assessments provide “a useful contribution to understanding the distribution and possible magnitude of marine and hydrokinetic energy sources in the United States\DIFdelbegin \DIFdel{”,}\DIFdelend \DIFaddbegin \DIFadd{,” }\DIFaddend but also detailed several specific technical critiques \citep{nationalresearchcouncilEvaluationDepartmentEnergy2013}. In particular, the NAS review \DIFdelbegin \DIFdel{revealed }\DIFdelend \DIFaddbegin \DIFadd{pointed out }\DIFaddend that the method used in the wave resource assessment did not account for wave direction, and therefore it “has the potential to double-count a portion of the wave energy” \citep{nationalresearchcouncilEvaluationDepartmentEnergy2013}. Over the past decade, wave resource assessment studies have been performed based on different methodologies; some works followed the EPRI 2011 methodology \DIFdelbegin \DIFdel{\mbox{%DIFAUXCMD
\citep[e.g., ]{kumarWaveEnergyResource2015}}\hskip0pt%DIFAUXCMD
}\DIFdelend \DIFaddbegin \DIFadd{\mbox{%DIFAUXCMD
\citep[e.g., ][]{kumarWaveEnergyResource2015}}\hskip0pt%DIFAUXCMD
}\DIFaddend , while others did account for directionality \DIFdelbegin \DIFdel{\mbox{%DIFAUXCMD
\citep{gunnQuantifyingGlobalWave2012, garcia-medinaWaveResourceAssessment2014, regueroGlobalWavePower2015}}\hskip0pt%DIFAUXCMD
}\DIFdelend \DIFaddbegin \DIFadd{\mbox{%DIFAUXCMD
\citep{gunnQuantifyingGlobalWave2012, regueroGlobalWavePower2015}}\hskip0pt%DIFAUXCMD
}\DIFaddend , but most assessments at large scales avoid the directionality issue, resulting in a reconnaissance-level assessment that does not directly quantify total power available across the region \citep{iglesiasWaveEnergyPotential2009, neillWavePowerVariability2013, sierraWaveEnergyResource2013, robertsonCharacterizingShoreWave2014, alonsoWaveEnergyResource2015, zhengAssessingChinaSea2013}. The use of inconsistent methodologies for wave resource assessment makes it difficult to compare studies, puts the industry’s credibility at risk, and does not provide policy makers the accurate information they need. 

Furthermore, there are two additional concerns \DIFdelbegin \DIFdel{on }\DIFdelend \DIFaddbegin \DIFadd{with }\DIFaddend the approach used in EPRI 2011\DIFdelbegin \DIFdel{wave resource assessment}\DIFdelend : 1) the domains of the resource assessments are often arbitrary and are not necessarily consistent with relevant political boundaries, and 2) the methods only account for wave energy arriving at the edge of the boundary and not \DIFdelbegin \DIFdel{for }\DIFdelend waves generated within the domain, which in this study we refer to as the local resource.
\DIFdelbegin \DIFdel{In addition, wave energy technology is still at an early-stage of technology maturity with a wide-range of device archetypes under active development \mbox{%DIFAUXCMD
\citep{babaritOceanWaveEnergy2017}}\hskip0pt%DIFAUXCMD
. Thus, improvement of device agnostic resource characterization methodology will be beneficial for the industry.
}\DIFdelend 

The primary objective of this work is to introduce a methodology that provides an accurate regional theoretical wave resource estimate for any scale and boundary geometry. The method developed in this study will bring clarity to the field of wave resource assessment and thereby facilitates comparison of regional total wave energy potentials. The secondary objective of this work is to provide a refined estimate of the U.S. wave resource based on this new method\DIFdelbegin \DIFdel{. These objectives are accomplished by discussing the details of regional resource assessment and presenting our proposed approach (Section \ref{sec:method}). Section \ref{sec:results} presents the results of this approach applied to each region of the U.S. coastline along with an analysis of the resource in the regional and temporal context. Summary, concluding remarks, and a view toward how to unify wave energy site and resource assessments are presented in Section \ref{sec:conclusion}}\DIFdelend .
