\section{Introduction}

Over the last decade global interest in wave energy has grown
considerably, with the Ocean Energy Systems executive committee
setting a goal of 300GW of installed ocean energy capacity by 2050
\citep[]{huckerbyInternationalVisionOcean2017}. This has been driven
by concerns about greenhouse-gas driven climate change, carbon-based
energy price volatility, and a growing emphasis on the value of
diversifying energy sources. Wave energy is seen as particularly
valuable for its predictability and reliability on timescales of hours
to a few days \citep{parkinsonIntegratingOceanWave2015}, and for the
fact that the resource is concentrated along coastlines where large
fractions of the world's populations live. Furthermore, the
U.S. Department of Energy (DOE) has recently initiated a ``Powering the Blue
Economy'' initiative designed to support the development of
technologies that provide power to unique market applications such as
desalination, ocean sensing, aquaculture, and coastal community
resiliency \citep{PBE_REPORT}. Wave energy technology is still at an
early-stage of technology development with a wide-range of device
archetypes under active development
\citep[]{babaritOceanWaveEnergy2017}.

The 2011 DOE-funded U.S. wave resource assessment provided the first
comprehensive estimate of the nation’s wave energy resource, and has
been an important reference for motivating private and public
investment in wave energy research ever since
\citep[][hereafter `EPRI 2011']{EPRIwaveresource2011}. A few years later, the International Electrotechnical Commission's Technical Committee on
Marine Energy (TC114) published the first edition of a wave
resource assessment technical specification \citep[][hereafter, 62600-101]{internationalelectrotechnicalcommissionPart101Wave2015}. The stated objective of 62600-101 is to provide ``a uniform
methodology that will ensure consistency and accuracy in the
estimation, measurement, and analysis of the wave energy resource at
sites that could be suitable for the installation of Wave Energy
Converters (WECs), together with defining a standardised methodology
with which this resource can be described.''

It is important to clarify that the EPRI 2011 resource assessment, and the methodology defined in IEC 62600-101 occupy distinct and complimentary branches of resource assessment:
\begin{enumerate}
\item {\bf \em Regional resource assessments} \note{is there a different word (not "regional") we can use that doesn't conflict with the "scale" axis of the Table \ref{tab:scale-type}? Perhaps "jurisdictional"?} seek to quantify the total power available over the region of interest in units of power (e.g., kW, GW, or TWh/yr). 
They are of value to policy makers who need a direct and clear assessment of the magnitude of resource in their region for comparison to other resource options, or for comparison to other regions. They are typically calculated over relatively large areas (e.g., O(1,000 km$^2$) all the way up to global scales). The EPRI 2011 report is a regional resource assessment for wave energy, \citep[e.g., ][]{EPRIwaveresource2011,gunnQuantifyingGlobalWave2012,regueroGlobalWavePower2015,motaWaveEnergyPotential2014}. 
\item {\bf \em Site resource assessments} provide the resource statistics that are useful for evaluating a site’s power production potential. They quantify the resource opportunity in terms of resource statistics such as wave height, wave period, wave direction, omni-directional wave power (in kW/m), the joint-probability distribution of wave-height and wave-period, and the directional wave spectrum. They are of value to project developers who use the statistics to quantify the opportunity for developing a project at that specific site. The IEC 62600-101 technical specification provides a detailed methodology for conducting site assessments, computing these statistics, and it also provides a consistent format for presenting the results. By using the 62600-101, developers and other parties can compare sites and projects efficiently and with confidence in the accuracy and consistency of the underlying methods. 
\end{enumerate}

The distinction between these two branches of resource assessment is subtle but important. Site resource assessment is motivated by the needs of project developers to understand the opportunity at specific sites, and for specific projects involving specific technologies, and several works follow the IEC 62600-101 methodology \citep[e.g., ][]{zhengAssessingChinaSea2013, neillWavePowerVariability2013, sierraWaveEnergyResource2013, robertsonCharacterizingShoreWave2014, yangCharacteristicsVariabilityNearshore2020, lokuliyana_sri_2020}. 
That branch of resource assessment is important because the certified project assessments that follow from the IEC TC114 methods are expected to improve investor and financier confidence, and thereby reduce the cost of marine energy.
This work, however, is focused primarily on the topic of regional resource assessment, because it is also important to ensure that regional resource assessments -- which motivate public investment in the technology -- are based on methodologies that have widespread scientific consensus. The first is motivated by the needs of policy-makers to understand the opportunity across large spatial scales.

It is also worth noting that these two branches of resource assessment are different from the three `classes' of resource assessment defined in IEC 62600-101 (reconaissance, feasibility, and design). Those classes are defined primarily in terms of level-of-accuracy (low, medium, and high, respectively) at distinct spatial scales, and they are all essentially site assessments in the sense that they generate the statistics of interest to project developers.
The IEC 62600-101 standard does not provide definitive guidance on regional resource assessment, but it does include the language ``the effects of [WEC arrays] on wave propagation should be included in the numerical model'', which ensures that the results of that work is consistent with regional resource assessments like this one.

Another important distinction within the field of resource assessment is related to the technology considerations in and spatial constraints on resource assessments. In this space IEC TC114 provides clear and informative definitions \citep{internationalelectrotechnicalcommissionPartTerminology2020}:
\begin{itemize}
  \item the {\it theoretical resource} is the ``energy available in the resource''
  \item the {\it technical resource} is the ``proportion of the theoretical resource that can be captured using existing technology options without consideration of external constraints''
  \item the {\it practical resource} is the ``proportion of the technical resource that is available after consideration of external constraints''
\end{itemize}

The theoretical resource is a basis for the other two flavors of resource assessment, as it indicates the total power that could be extracted from the resource according to laws of physics and assuming perfectly efficient technology. In the case of wave energy, the energy {\em available} is equivalent to the energy {\em contained in} the wave field because it is theoretically possible to extract all of the energy from the waves as originally demonstrated by the ``Salter Duck'' \citep{salterRecentProgressDucks1980} \note{We probably need to say something about Budal limits here?!}. This is in contrast to the ``Betz limit'' of wind energy, which states that a wind turbine cannot extract more than 59.3 \% of the kinetic energy in the wind.

Real technologies, of course, are not perfectly efficient. Wave energy converters often reflect some wave energy, and generate some amount of turbulence in the vicinity of the device that is lost to heat. And of course there are losses (to heat) in converting mechanical energy to electricity and transmitting that electricity to the grid. An estimate of the ``technical resource'' accounts for these factors in order to provide decision makers (whether they are project developers, or policy makers) an estimate of the opportunity in terms of real technologies available today. In site resource assessments, this is typically accomplished by combinging the joint-probability distribution (e.g., from IEC 62600-101) with a specific technology's power matrix (e.g., from IEC 62600-100) to estimate a project's annual energy production (AEP, in units of power).
EPRI 2011 provides a regional estimate of the U.S. technical resource based on reasonable assumptions about device conversion efficiency and device spacing.

The practical resource takes another step in considering what is realistic by accounting for `external constraints'. These constraints account for the realistic ocean and land-use factors that cannot be quantified purely in terms of physical or technical constrains, such as: permitting, alternate ocean uses, and other types of restrictions that limit the placement and energy extraction of projects. At regional scales, practical resource assessment typically involves ad-hoc ``fraction of the technical resource'' assumptions. This is because a detailed accounting of external constraints at regional scales is a massive undertaking, and is particularly challenge when wave energy technologies are still in their infancy (i.e., stakeholders want to know the details of what will be installed). At the site-specific scale, a project is often designed to maximize the energy production at a site within the external constraints that exist there, which is a defacto practical resource assessment.

In 2013 the National Academy of
Sciences published a review of the DOE’s regional marine energy resource
assessments, including: tidal, ocean current, in-stream river, ocean
thermal energy, and the EPRI 2011 wave energy resource assessment
\citep{nationalresearchcouncilEvaluationDepartmentEnergy2013}.  The
National Academy generally found that the resource assessments provide
“a useful contribution to understanding the distribution and possible
magnitude of marine and hydrokinetic energy sources in the United
States,” but also had specific technical critiques of each assessment,
and of the comparability of the results between each of the
assessments.

In the case of wave energy, the National Academy review points out that
because the method used did not account for wave direction, it ``has the
potential to double-count a portion of the wave energy''. Furthermore, discussion with several leading wave energy experts reveals two additional key criticisms of existing wave resource assessment methods: 1) that the domain of the resource assessments are often arbitrary in that they are not necessarily consistent with the relevant political boundaries that they are meant to be applicable to, and 2) that the methods only account for wave energy arriving at the edge of the boundary and do not account for waves generated within the boundary itself.

These lingering critiques have resulted in a ``define your own method'' approach to regional wave resource assessment. Some works follow the EPRI 2011 methodology \citep[e.g., ]{kumarWaveEnergyResource2015}, while others account for directionality \citep[e.g., ]{gunnQuantifyingGlobalWave2012, regueroGlobalWavePower2015, hemerRevisedAssessmentAustralia2017}, but most assessments at large scales seem to avoid the issue and instead deliver a reconnaissance-level assessment that does not directly quantify total power available across the region \citep[e.g.,]{robertsonCharacterizingShoreWave2014, sierraWaveEnergyResource2013, zhengAssessingChinaSea2013, neillWavePowerVariability2013, alonsoWaveEnergyResource2015}.
This makes it difficult to make apples-to-apples comparisons, puts the industry's credibility at risks, and -- perhaps most importantly -- doesn't provide policy makers with the kind of information they need. The primary objective of this work is to address the critiques and questions in the previous paragraph directly, and thereby bring clarity to the topic of region-scale theoretical resource assessment. In time, we hope that this clarity serves as a basis for consensus on the subject so that researchers can deliver assessments that are consistent, and then move on to other important research areas -- including technical and practical resource assessment. The secondary objective of this work is to provide a refined estimate of
the U.S. wave resource based on this new method, and new model
results. 

These objectives are accomplished by first discussing the details of regional theoretical resource assessment and presenting our proposed approach (section 2). In section 3, we present the results of this approach applied to each region of the U.S. coastline in comparison to alternate methods that have been used. Section 4 provides a detailed discussion of the results, \note{IS THIS IN HERE? including a justification for the proposed approach, and a summary of how the proposed approach can be applied in different scenarios}. We conclude (section 5) with a summary of results, the proposed methodology, and a view toward how to unify wave energy site and resource assessment.


%%% Local Variables:
%%% TeX-master: "wave_res"
%%% End:
