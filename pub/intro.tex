\section{Introduction}

Though wave energy technology is still at an
early-stage of technology development, interest in wave energy has grown
considerably over the last decade
\citep[]{babaritOceanWaveEnergy2017}. For example, the Ocean Energy Systems executive committee has set a goal of 300GW of installed ocean energy capacity by 2050 \citep[]{huckerbyInternationalVisionOcean2017}. Wave energy is seen as particularly
valuable for its predictability and reliability on timescales of hours
to a few days \citep{parkinsonIntegratingOceanWave2015}, and for the
fact that the resource is concentrated along coastlines where large
fractions of the world's populations live. Furthermore, the
U.S. Department of Energy (DOE) has launched the ``Powering the Blue
Economy'' initiative designed to support the development of
technologies that provide power to unique market applications such as
desalination, ocean sensing, aquaculture, maritime shipping, and coastal community
resiliency \citep{livecchiPoweringBlueEconomy2019}. 

There have been several notable wave resource assessments over the last two decades that have gradually improved the accuracy and spatial coverage of our understanding of the U.S. wave energy opportunity \citep[]{EPRIwaveresource2011, garcia-medinaWaveResourceAssessment2014, bedardOceanWaveEnergy2005}. Throughout this time, however, there have been ongoing debates about the correct methodology for quantifying the wave energy opportunity. While a clear consensus-based methodology for wave resource site-assessment has been established, regional wave resource assessments have utilized disparate methodologies, which can lead to skepticism and confusion \citep[]{robertsonCharacterizingShoreWave2014, gunnQuantifyingGlobalWave2012, hughesNationalscaleWaveEnergy2010, hemerRevisedAssessmentAustralia2017, nationalresearchcouncilEvaluationDepartmentEnergy2013}.

The 2011 DOE-funded U.S. wave resource assessment provided the first
comprehensive estimate of the nation’s wave energy resource, and has
been an important reference for motivating private and public
investment in wave energy research ever since
\citep[][hereafter `EPRI 2011']{EPRIwaveresource2011}. A few years later, the International Electrotechnical Commission's Technical Committee on
Marine Energy (TC114) published the first edition of a wave
resource assessment technical specification \citep[][hereafter, 62600-101]{internationalelectrotechnicalcommissionPart101Wave2015}. The stated objective of 62600-101 is to provide ``a uniform
methodology that will ensure consistency and accuracy in the
estimation, measurement, and analysis of the wave energy resource at
sites that could be suitable for the installation of Wave Energy
Converters (WECs), together with defining a standardised methodology
with which this resource can be described.''

In 2013 the National Academy of Sciences (NAS) published a review of the DOE’s regional marine energy resource assessments, which included the EPRI 2011 wave energy resource assessment. NAS found that in general the resource assessments provided “a useful contribution to understanding the distribution and possible magnitude of marine and hydrokinetic energy sources in the United States,” but also detailed specific technical critiques of each assessment. In the case of wave energy, the NAS review stated that because the method used did not account for wave direction, it “has the potential to double-count a portion of the wave energy”. 

Furthermore, discussions with several leading wave energy experts revealed two additional key criticisms of existing wave resource assessment methods: 1) that the domain of the resource assessments are often arbitrary and are not necessarily consistent with relevant political boundaries, and 2) that the methods only account for wave energy arriving at the edge of the boundary and not for waves generated within the boundary.

These lingering critiques have resulted in a “define your own method” approach to regional wave resource assessment. Some works follow the EPRI 2011 methodology, while others do account for directionality, but most assessments at large scales avoid the issue resulting in a reconnaissance-level assessment that does not directly quantify total power available across the region. This makes it difficult to make apples-to-apples comparisons, puts the industry’s credibility at risk, and does not provide policy makers with the kind of information they need. 

The primary objective of this work is to address the critiques and questions in the previous paragraph directly, and thereby bring clarity to the topic of region-scale theoretical resource assessment. In time, we hope that this clarity serves as a basis for consensus on the subject so that researchers can deliver assessments that are consistent, and then move on to other important research areas – including technical and practical resource assessment. The secondary objective of this work is to provide a refined estimate of the U.S. wave resource based on this new method and updated model outputs.

%%% Local Variables:
%%% TeX-master: "wave_res"
%%% End:
